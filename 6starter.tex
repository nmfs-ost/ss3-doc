\hypertarget{StarterFile}{}
\section[Starter File]{\protect\hyperlink{StarterFile}{Starter File}}

\hypertarget{ManualFormat}{}
\subsection[Reading the Manual's format]{\protect\hyperlink{ManualFormat}{Reading the Manual's format}}
SS3 begins by reading the file \texttt{starter.ss}. The starter file contains need information on the names of the control and data files, run conditions, and output specifications. The term COND appears in the ``Typical Value'' column of this documentation (it does not actually appear in the model files), it indicates that the following section is omitted except under certain conditions, or that the factors included in the following section depend upon certain conditions. In most cases, the description in the definition column is the same as the label output to the ss\_new files.

\hypertarget{FTerminology}{}
\subsection[Terminology for Fishing Mortality, $F$]{\protect\hyperlink{FTerminology}{Terminology for Fishing Mortality, $F$}}
Here we introduce some terminology related to fishing mortality, $F$. This will provide context for some of the quantities that will be read from the starter file and used throughout the document

$f$ is fleet.

$t$ is a time step; continuous across years $y$ and seasons $s$; equivalent to year if only 1 season.

$a$ is age.

$s_{t,f,a}$ is age-specific selectivity for a fleet. If selectivity is length-specific, then age-specific selectivity due to length-selectivity is calculated as the dot product across length bins of length selectivity and the normal (or log-normal) distribution of length-at-age. If selectivity is both length- and age-based, which is an entirely normal concept in SS3, then age selectivity due to length selectivity is calculated first, then multiplied by the direct age selectivity. This compound age selectivity is used in the mortality calculations and is reported as Asel2 in report:32 of \texttt{Report.sso}. Selectivity can be sex-specific, and different growth morphs and platoons can have different age-selectivity due to the effect of length-selectivity on their unique size-at-age. This added dimension, $g$, for biological group is not included in the nomenclature here but exists in all the SS3 calculations.

$F_{t,f,a}$ is fishing mortality at age for fleet $f$. There is no subscript for area because each fleet is defined to operate in only one area.

$F_{t,f}'$ is a fleet's fishing mortality for the age that has selectivity equal to 1.0. This is also termed F' or $\text{full\_}F$ in the SS3 system. If your model is using parameters for $F$, then the parameter values are for the $F'$. Note that some selectivity curves, like double normal, are explicit about having a maximum of 1.0. But other curves like logistic and combinations of length-selectivity and growth, may produce an age-selectivity curve that never reaches 1.0 and time-varying non-parametric selectivity will produce values > 1.0 routinely. In all cases, the resultant $F_{t,f,a}$ comes from $F_{t,f}' * s_{t,f,a}$, so the range of the $F'$ compensates for the scale of the $s$.

Apical selectivity is the maximum age-specific selectivity and is not explicit in any internal calculation in SS3, it is just for reporting. If selectivity has a maximum value of 1.0, then $\text{apical\_}F$ and $\text{full\_}F$ are identical.

Fully-selected age range is not explicitly used in SS3, especially because SS3 applications routinely have multiple fleets with different selectivity patterns that may have little overlap.

Fbar is the average $F$ over a user specified range of ages, implicitly the fully-selected range for the total $F$ from all the fleets. Some SS3 output options will display Fbar.

$\text{Annual\_}F$ is essentially the same as Fbar and is an output quantity.

$F\text{\_std}$ is an output quantity that may be based on $\text{annual\_}F$ or other calculated quantities like exploitation rate. Importantly, the output values of $F\text{\_std}$ may be presented as a ratio relative to an equivalent benchmark (reference point) quantity; e.g., $F / F_{MSY}$. Further, the variance of $F\text{\_std}$ will be calculated and output.

$C_{t,f}$ is fleet-specific catch in a time step.

$B_{t,f}$ is fleet specific available biomass, e.g., total biomass filtered by fleet-specific age selectivity, $s_{t,f,a}$. Note that this is not adjusted by the $max(s_{t,f,a})$.


{
\setlength\extrarowheight{4pt}
\begin{landscape}
\hypertarget{StarterOptions}{}
\subsection[Starter File Options (\texttt{starter.ss})]{\protect\hyperlink{StarterOptions}{Starter File Options (\texttt{starter.ss})}}	

\begin{longtable}{p{1.5cm} p{7.2cm} p{12.3cm}} 

 \hline
 \textbf{Value} & \textbf{Options} & \textbf{Description} \TBstrut\\ 
 \hline
 \endfirsthead
 
 \hline
 \textbf{Value} & \textbf{Options} & \textbf{Description} \TBstrut\\ 
 \hline
 \endhead
 
 \hline
 \endfoot
 
 \hline
 \multicolumn{3}{c}{\textbf{End of Starter File}} \Tstrut\Bstrut\\
 \hline
 \endlastfoot

 \#C this is a starter comment & Must begin with \#C then rest of the line is free form & All lines in this file beginning with \#C will be retained and written to the top of several output files \Tstrut\\
		
 \hline
 \texttt{data\_file.dat} &  & File name of the data file \Tstrut\\
		
 \hline
 \texttt{control\_file.ctl} &  & File name of the control file \Tstrut\\
   
 \hline		
 0 & Initial Parameter Values: & \multirow{1}{12.3cm}[-0.25cm]{\parbox{12.3cm}{Do not set equal to 1 if there have been any changes to the control file that would alter the number or order of parameters stored in the \texttt{ss3.par} file. Values in \texttt{ss3.par} can be edited, carefully. Do not run \texttt{ss\_trans.exe} from a \texttt{ss3.par} from v.3.24.}}\Tstrut\\
 & 0 = use values in control file; and &  \\
 & 1 = use \texttt{ss3.par} after reading setup in the control file. & \\
		
 \hline
 1 & Run display detail: &  \multirow{2}{12.3cm}[-0.25cm]{\parbox{12.3cm}{With option 2, the display shows value of each negative log likelihood component for each iteration, and it displays where crash penalties are created}} \Tstrut\\
   & 0 = none other than \gls{admb} outputs; & \\
   & 1 = one brief line of display for each iteration; and & \\
   & 2 = fuller display per iteration. & \\
		  
 \hline
 1 & Detailed age-structure report: & \multirow{1}{12.3cm}[-0.15cm]{\parbox{12.3cm}{Option 0 will forgo the writing of the Report file, but the ss\_summary file will be written that has minimal derived and estimated quantities. This is a useful option for some data-limited assessment approaches (e.g., \href{https://github.com/chantelwetzel-noaa/XSSS}{\texttt{xsss}} or \href{https://github.com/shcaba/SSS}{\texttt{sss}}). Option 1 will write out the full Report file. Option 2 will write out select items in the Report file and will omit some more detailed sections (e.g., numbers-at-age).}} \Tstrut\\
   & 0 = minimal output for data-limited methods; & \\
   & 1 = include all output (with \texttt{wtatage.ss\_new}); & \\
   & 2 = brief output, no growth;  and &  \\	
   & 3 = custom output & \\
\hline
%  \pagebreak

 \multicolumn{2}{l}{COND: Detailed age-structure report = 3} & \multirow{4}{12.3cm}[-0.25cm]{\parbox{12.3cm}{Custom report options: First value: -100 start with minimal items or -101 start with all items; Next Values: A list of items to add or remove where negative number items are removed and positive number items added, -999 to end. The \hyperlink{custom}{reporting numbers} for each item that can be selected or omitted are shown in the Report file next to each section key word.}} \Tstrut\\
 \multicolumn{1}{r}{-100} & & \\
 \multicolumn{1}{r}{  -5} & & \\
 \multicolumn{1}{r}{   9} & & \\
 \multicolumn{1}{r}{  11} & & \\
 \multicolumn{1}{r}{  15} & & \\
 \multicolumn{1}{r}{-999} & & \Bstrut\\
		 
 \hline
 0 & Write 1st iteration details: & \multirow{2}{12.3cm}[-0.25cm]{\parbox{12.3cm}{This output is largely unformatted and undocumented and is mostly used by the developer.}} \Tstrut\\
   & 0 = omit; and & \\
   & 1 = write detailed intermediate calculations to \texttt{echoinput.sso} during first call. & \Bstrut\\

 \hline
 0 & Parameter Trace: & \multirow{1}{12.3cm}[-0.25cm]{\parbox{12.3cm}{This controls the output to \texttt{parmtrace.sso}. The contents of this output can be used to determine which values are changing when a model approaches a crash condition. It also can be used to investigate patterns of parameter changes as model convergence slowly moves along a ridge. In order to access parameter gradients option 4 should be selected which will write the gradient of each parameter with respect to each likelihood component}} \Tstrut\\
   & 0 = omit; & \\
   & 1 = write good iteration and active parameters; & \\
   & 2 = write good iterations and all parameters; & \\
   & 3 = write every iteration and all parameters; and & \\
   & 4 = write every iteration and active parameters. & \Bstrut\\
 \hline
 
%  \pagebreak
 1 & Cumulative Report: & \multirow{1}{12.3cm}[-0.25cm]{\parbox{12.3cm}{Controls reporting to the file \texttt{Cumreport.sso}. This cumulative report is most useful when accumulating summary information from likelihood profiles or when simply accumulating a record of all model runs within the current subdirectory}} \Tstrut\\
   & 0 = omit;  & \\
   & 1 = brief; and & \\
   & 2 = full. & \Bstrut\\
	 
 \hline
 1 & Full Priors: & \multirow{1}{12.3cm}[-0.25cm]{\parbox{12.3cm}{Turning this option on (1) adds the log likelihood contribution from all prior values for fixed and estimated parameters to the total negative log likelihood. With this option off (0), the total negative log likelihood will include the log likelihood for priors for only estimated parameters.}} \Tstrut\\
   & 0 = only calculate priors for active parameters; and &	\\
   & 1 = calculate priors for all parameters that have a defined prior. & \Bstrut\\
	     
 \hline
 1 & Soft Bounds: & \multirow{1}{12.3cm}[-0.25cm]{\parbox{12.3cm}{This option creates a weak symmetric beta penalty for the selectivity parameters. This becomes important when estimating selectivity functions in which the values of some parameters cause other parameters to have negligible gradients, or when bounds have been set too widely such that a parameter drifts into a region in which it has negligible gradient. The soft bound creates a weak penalty to move parameters away from the bounds.}}\Tstrut\Bstrut\\
   & 0 = omit; and & \\
   & 1 = use. & \\
   & & \\
   & & \\
   & & \\

 \pagebreak
 \hline
 1 & Number of Data Files to Output: & \multirow{5}{12.3cm}[-0.25cm]{\parbox{12.3cm}{All output files are sequentially output to \texttt{data\_echo.ss\_new} and need to be parsed by the user into separate data files. The output of the input data file makes no changes, retaining the order of the original file. Output files 2-N contain only observations that have not been excluded through use of the negative year denotation, and the order of these output observations is as processed by the model. At this time, the tag recapture data is not output to \texttt{data\_echo.ss\_new}. As of v.3.30.19, the output file names have changed; a separate file is created for the echoed data (\texttt{data\_echo.ss\_new}), the expected data values given the model fit (\texttt{data\_expval.ss}), and any requested bootstrap data files (\texttt{data\_boot\_x.ss} where x is the bootstrap number). In versions before v.3.30.19, these outputs were printed to a single file called \texttt{data.ss\_new}.}} \Tstrut\Bstrut\\
   & 0 = none; As of v.3.30.16, none of the \texttt{.ss\_new} files will be produced; & \Bstrut\\
   & 1 = output an annotated replicate of the input data file; & \Tstrut\Bstrut\\
   & 2 = add a second data file containing the model's expected values with no added error; and & \Tstrut\Bstrut\\
   & 3+ = add N-2 parametric bootstrap data files. & \Tstrut\\
   & & \\

 \hline
 %\pagebreak
 8 & Turn off estimation: & \multirow{1}{12.3cm}[-0.25cm]{\parbox{12.3cm}{The 0 option is useful for (-1) quickly reading in a messy set of input files and producing the annotated \texttt{control.ss\_new} and \texttt{data\_echo.ss\_new} files, or (0) examining model output based solely on input parameter values. Similarly, the value option allows examination of model output after completing a specified phase. Also see usage note for restarting from a specified phase.}} \Tstrut\\
   & -1 = exit after reading input files; & \\
   & 0 = exit after one call to the calculation routines and production of sso and ss\_new files; and & \\
   & <positive value> = exit after completing this phase. & \Bstrut\\	  
	     
 \hline
 1000 & \gls{mcmc} burn interval & Number of iterations to discard at the start of a \gls{mcmc} run. \Tstrut\Bstrut\\
	   
 \hline
 %\pagebreak
 200 & \gls{mcmc} thin interval & Number of iterations to remove between the main period of the \gls{mcmc} run. \Tstrut\\
	
 \pagebreak
%  \hline 
 0.0 & \hyperlink{Jitter}{Jitter:} & \multirow{1}{12.3cm}[-0.25cm]{\parbox{12.3cm}{The jitter function has been revised with v.3.30. Starting values are now jittered based on a normal distribution with the $pr(P_{MIN}) = 0.1\%$ and the $pr(P_{MAX}) = 99.9\%$. A positive value here will add a small random jitter to the initial parameter values. When using the jitter option, take care when defining the low and high bounds for parameter values and particularly -999 or 999 should not be used to define bounds for estimated parameters.}} \Tstrut\Bstrut\\ 
	 & 0 = no jitter done to starting values; and & \Bstrut\\
	 & > 0 starting values will vary with larger jitter values resulting in larger changes from the parameter values in the control or par file. & \Bstrut\\
	
 \hline
 -1 & \Gls{sd} Report Start: & \Tstrut\\
    & -1 = begin annual \gls{sd} report in start year; and & \\
    & <year> = begin \gls{sd} report this year. & \Bstrut\\
	      
 \hline
%  \pagebreak
 -1 & \gls{sd} Report End: & \Tstrut\\
    & -1 = end annual \gls{sd} report in end year; & \\
    & -2 = end annual \gls{sd} report in last forecast year; and & \\
    & <value> = end \gls{sd} report in this year. & \Bstrut\\
	   
 \hline
 2 & Extra \gls{sd} Report Years: & \multirow{1}{12.3cm}[-0.25cm]{\parbox{12.3cm}{In a long time series application, the model variance calculations will be smaller and faster if not all years are included in the \gls{sd} reporting. For example, the annual \gls{sd} reporting could start in 1960 and the extra option could select reporting in each decade before then.}} \Tstrut \Bstrut\\
   & 0 = none; and & \\
   & <value> = number of years to read. & \Bstrut\\

 %\pagebreak 
 \hline  
 \multicolumn{3}{l}{COND: If Extra \gls{sd} report years > 0} \Tstrut\\

 %\pagebreak
 \hline
 \multicolumn{1}{r}{1940 1950} & \multirow{1}{12.3cm}[-0.25cm]{\parbox{12.3cm}{Vector of years for additional \gls{sd} reporting. The number of years needs to equal the value specified in the above line (Extra \gls{sd} Report Years).}} \\
  & & \\

 \pagebreak
%\hline
 0.0001 & Final convergence & \multirow{1}{12.3cm}[-0.25cm]{\parbox{12.3cm}{This is a reasonable default value for the change in log likelihood denoting convergence. For applications with much data and a large total log likelihood value, a larger convergence criterion may still provide acceptable convergence.}} \Tstrut\Bstrut\\
   & & \Bstrut\\
   & & \Bstrut\\
	%  & & \\ 
 
 \hline
 0 & Retrospective year: & \multirow{1}{12.3cm}[-0.25cm]{\parbox{12.3cm}{Adjusts the model end year and disregards data after this year. May not handle time varying parameters completely.}} \Tstrut\\
   & 0 = none; and & \\
   & -x = retrospective year relative to end year. & \Bstrut\\
  
 \hline
 0 & Summary biomass min age & \multirow{1}{12.3cm}[-0.25cm]{\parbox{12.3cm}{Minimum integer age for inclusion in the summary biomass used for reporting and for calculation of total exploitation rate.}} \Tstrut\\
   & & \\ 

 \hline
%  \pagebreak
 1 & Depletion basis: & \multirow{1}{12.3cm}[-0.25cm]{\parbox{12.3cm}{Selects the basis for the denominator when calculating degree of depletion in reproductive output (a.k.a., \gls{ssb}). The calculated values are reported to the \gls{sd} report relative to a fraction, X, of a comparable quantity calculated in the benchmark section or elsewhere.}} \Tstrut\\
   & & \\
   & & \\
   & 0 = skip; & \Tstrut \\
   & 1 = $X*B_{0}$; & Relative to virgin spawning biomass. This option will affect interpretation of biomass target in the \texttt{forecast.ss} file \\
   & 2 = $X*B_{MSY}$; & Relative to spawning biomass that achieves \gls{msy}. \\
   & 3 = $X*B_{styr}$; and & Relative to model start year spawning biomass. \\
   & 4 = $X*B_{endyr}$. & Relative to spawning biomass in the model end year. \\
   & 5 = $X*Dynamic~B_{0}$ & Relative to the calculated dynamic $B_{0}$. \\
   & 6 = $\mathtt{SSB\_unf\_bmark}$ & Relative to unfished spawning biomass in the benchmark. \\
   & use tens and hundreds digits to invoke multi-year trailing average & \\
   & append 0.1 to invoke ln(ratio) & \Bstrut\\
  
 \hline
 1.0 & Fraction (X) for depletion denominator & Value for use in the calculation of the ratio for $SB_{y}/(X*SB_{0})$. \Tstrut\Bstrut\\

 \hline
%  \pagebreak
 1 & \Gls{spr} report scaling: & \multirow{1}{12.3cm}[-0.25cm]{\parbox{12.3cm}{\gls{spr} is the equilibrium \gls{ssb} per recruit that would result from the current year's F-at-age. The quantities identified by 1, 2, and 3 here are calculated in the benchmarks section. Then the one specified here is used as the selected denominator in a ratio with the annual value of (1 - \gls{spr}). This ratio (and its variance) is reported to the \gls{sd} report output for the years selected above in the \gls{sd} report year selection.}} \Tstrut\\
   & 0 = skip; & \\
   & 1 = use $1-SPR_{TARGET}$; & \\
   & 2 = use $1-SPR$ at $MSY$; & \Tstrut\\
   & 3 = use $1-SPR$ at $B_{TARGET}$; and &  \Tstrut\\
   & 4 = no denominator, so report actual $1-SPR$ values. & \\
   & 5 = use $SPR$ & \\
  
 \pagebreak
\hline 
 4 & $F\text{\_std}$ reporting units: & \multirow{1}{12.3cm}[-0.25cm]{\parbox{12.3cm}{An additional proxy for fishing intensity is based on the fraction of the population that is caught. As with SPR, the selected quantity will be calculated annually and in the benchmarks section. The ratio of the annual value to the selected (see $F$ report basis below) benchmark value is reported to the \gls{sd} report vector as $F\text{\_std}$. Options 1 and 2 ignore details of age-structure and are simply based on annual exploitation rate across all areas. If most catch occurs in one area and there is little movement between areas, this ratio is not informative about the $F$ in the area where the catch occurs. Option 3 is a simple sum of the full $F$'s by fleet, so may provide non-intuitive results when there are multiple areas or seasons or when the selectivities by fleet do not have good overlap in age. Option 4 is a real $\text{annual\_}F$ calculated as a numbers weighted $F$ for a specified range of ages. The $F$ for each age is calculated as $Z-M$ where $Z$ and $M$ are each calculated as $ln(N_{t+1}/N_{t})$ with and without $F$ active, respectively. The numbers are summed over all biology morphs and areas for the beginning of the year, so subsumes any seasonal pattern.}} \Tstrut\Bstrut\\
   & 0 = skip; & \\
   & 1 = exploitation rate in biomass; & \\
   & 2 = exploitation rate in numbers; & \\
   & 3 = sum($F$'s by fleet); & \\
   & 4 = Fbar: numbers weighted $F$ for range of ages using $Z-M$ approach; and & \\
   & 5 = Fbar: unweighted average $F$ for range of ages. & \\
  %  & & \\
   & & \Bstrut\\
   & & \Bstrut\\
   & & \Bstrut\\
   & & Note that these $F$ statistics do not depend upon whether the $F$ approach uses mid-season exploitation rate (Pope's), or continuous $F$; nor whether the continuous $F$ is based on parameters or the hybrid calculation method. Read more about the $F$ method in the control file section of the report. For more information on $F$ reporting, see \hyperlink{FMortality}{Metrics for Fishing Mortality}. \Bstrut\\ 
  
 \hline
%  \pagebreak
 \multicolumn{2}{l}{COND: If $F\text{\_std}$ reporting $\geq$ 4} & \multirow{1}{12.3cm}[-0.25cm]{\parbox{12.3cm}{Specify range of ages. Upper age must be less than max age because of incomplete handling of the accumulator age for this calculation.}} \Tstrut\\
 \multicolumn{1}{r}{3 7}  & Age range if $F\text{\_std}$ reporting = 4. & \Tstrut\Bstrut\\

 \hline
 \pagebreak
 1 & $F\text{\_std}$ scaling: & \multirow{1}{12.3cm}[-0.25cm]{\parbox{12.3cm}{$F\text{\_std}$ is typically reported as a ratio to the value of an equivalent $F$ calculation that would occur at the benchmark level of fishing. Here the user selects the denominator for that ratio. This ratio can be presented as a multi-year trailing average in $F$ or as a ln(ratio). For example, 122.1 would do a 12-year trailing average of the ratio using $F_{MSY}$ and present the result as the ln(ratio).}} \Tstrut\\
   & 0 = not relative, report raw values; & \\
   & 1 = use $F\text{\_std}$ value relative to $SPR_{TARGET}$; & \\
   & 2 = use $F\text{\_std}$ value relative to $F_{MSY}$; and & \\
   & 3 = use $F\text{\_std}$ value relative to $F_{B_{TARGET}}$. & \\
   & use tens and hundreds digits to invoke multi-year averaged $F\text{\_std}$ & \\
   & append 0.1 to the integer to invoke ln(ratio) & \Bstrut\\

  \hline
  % \pagebreak
  0.01 & \gls{mcmc} output detail: & \multirow{1}{12.3cm}[-0.25cm]{\parbox{12.3cm}{Specify format of \gls{mcmc} output. This input requires the specification of two items; the output detail and a bump value to be added to the $ln(R_{0})$ in the first call to \gls{mcmc}. A bias adjustment of 1.0 is applied to recruitment deviations in the \gls{mcmc} phase, which could result in reduced recruitment estimates relative to the \gls{mle} when a lower bias adjustment value is applied. A small value, called the ``bump'', is added to the $ln(R_{0})$ for the first call to \gls{mcmc} in order to prevent the stock from hitting the lower bounds when switching from \gls{mle} to \gls{mcmc}. If you wanted to select the default output option and apply a bump value of 0.01 this is specified by 0.01 where the integer value represents the output detail and the decimal is the bump value.}} \Tstrut\Bstrut\\
  & 0 = default; & \\
  & 1 = output likelihood components and associated lambda values; & \\
  & 2 = write report for each mceval; and & \\		 
  & 3 = make output subdirectory for each \gls{mcmc} vector. & \Bstrut\\
  & & \\
  & & \Tstrut\Bstrut\\ 		 
  
  % \pagebreak
  \hline
  \raisebox{0.1\ht\strutbox}{\hypertarget{ALK}{0}} & \gls{alk} tolerance level & This effect is disabled in code, enter 0. \Tstrut\Bstrut\\

  % \pagebreak
  \hline  
  \multicolumn{2}{l}{COND: Seed Value (i.e., 1234)} & \multirow{1}{12.3cm}[-0.25cm]{\parbox{12.3cm}{Specify a seed for data generation. This feature is not available in versions prior to v.3.30.15 This is an optional input value allowing for the specification of a random number seed value. If you do not want to specify a seed, skip this input line and end the starter file with the check value (3.30).}} \Tstrut\Bstrut\\
  & & \\ 
  & & \Bstrut\\
  & & \\
  
  0 & \multirow{1}{12.3cm}[-0.25cm]{\parbox{12.3cm}{Compatibility flag implemented as of v.3.30.24. Use 0 for legacy or 1 for improved impact of time-varying biology on benchmark SRR calculations.}} \Tstrut\Bstrut\\
  & & \\

 \pagebreak
%  \hline
 \raisebox{0.1\ht\strutbox}{\hypertarget{Convert}{3.30}} & Model version check value. & \multirow{1}{12.3cm}[-0.25cm]{\parbox{12.3cm}{A value of 3.30 indicates that the control and data files are currently in v.3.30 format. A value of 999 indicates that the control and data files are in a previous v.3.24 version. The \texttt{\texttt{ss\_trans.exe}} executable should be used to convert the v.3.24 \texttt{\texttt{control.ss\_new}} and \texttt{\texttt{data\_echo.ss\_new}} files to the new format. All ss\_new files are in the v.3.30 format, so \texttt{\texttt{starter.ss\_new}} has v.3.30 on the last line. The mortality-growth parameter section has a new sequence and v.3.30 cannot read a \texttt{\texttt{ss3.par}} file produced by v.3.24 and earlier, so ensure that the read par file option at the top of the starter file is set to 0. The \hyperlink{ConvIssues}{Converting Files from Stock Synthesis v.3.24} section has additional information on model features that may impede file conversion.}} \Tstrut\Bstrut\\
     & & \\  
     & & \\  
	   & & \\
     & & \\
   	 & & \\
     & & \\  
     & & \\  
     & & \\

\end{longtable}
\end{landscape}
}
\restoregeometry





\pagebreak
