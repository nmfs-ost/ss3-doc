\section{Advanced SS Configuration Settings and Advice}

\hypertarget{TVpara}{}
\subsection{Using Time-Varying Parameters}

\myparagraph{Time-Varying Parameter Change from Earlier SS Versions}
The approach to time varying parameters was overhauled from SS v.3.24 to SS v.3.30. In SS v.3.24, the group of biology parameters (termed mortality-growth parameters) and the selectivity parameters used the same long parameter line approach, but each was implemented in its own section of the code. The spawner-recruitment parameters used short parameter lines and a different approach for linkage to an environmental variable and the R1 offset provided a limited type of block. The catchability parameters also used short parameter lines and had its own approach to doing environmental linkage and random deviations, but not blocks. Then finally, the tagging parameters had long parameter lines, but there was no code to interpret any time-varying info in those lines.

\myparagraph{Code Flow Version SS v.3.30}
In SS v.3.30, mortality-growth, selectivity, stock recruitment relationship, catchability, and tagging (soon but not as of v.3.30.16) base parameters all use long parameter lines and invoke environmental linkages, random deviations, blocks, and trends using identical syntax across parameters. Environmental linkages, random deviations, blocks/trends all can be applied to the same base parameter. Only blocks and trends are mutually exclusive, but any combined effect could be used together judiciously.  SS processes each time-varying parameter specification and creates a time-series of parameter values that are used as SS subsequently loops through years.

\myparagraph{Parameter Order}
 In SS v.3.30, the time varying input short parameter lines are re-organized such that all parameters that affect a base parameter are clustered together with block/trend first, then environmental, then deviation. For example, if mortality-growth (MG) base parameters 3 and 7 had time varying changes, the order would look like:

\begin{center}
	\begin{longtable}{p{5cm} p{10cm}}
		\hline
		MG base parameter 3 & Block parameter 3-1\Tstrut\\
		& Block parameter 3-2\\
		& Environmental link parameter 3-1\\
		& Deviation se parameter 3 \\
		& Deviation rho parameter 3 \Bstrut\\
		MG base parameter 7 & Block parameter 7-1 \\
		& Deviation se parameter 7 \\
		& Deviation rho parameter 7 \Bstrut\\
		\hline	 	                    
		
	\end{longtable}
\end{center}

\myparagraph{Link Functions} 
The functional form by which a time-varying parameter, P_t, changes a base parameter, P_{base}, is a link function:  $P_y=f(P_{base},P_t)$, where $P_y$ is the final parameter value in year $y$. Commonly used links are additive or multiplicative functions, but there are other possiblities. The link specifications in SS v.3.30 has been updated from SS v.3.24. Note that link functions (as of SS v 3.30.16) are specific to each type of time varying parameter (e.g., links codes for environmental links are different than the link codes for time blocks).

Another type of link in SS is between a model state variable, such as available biomass, and the expected value for a survey.  Typically, this is a simple proportional link taking one parameter, q, but the q power feature is essentially a 2 parameter link function. So, a parameter link function can change q over time, and a survey link function then uses the annual value of q to link the annual value of a state variable to the expected value for a survey. As SS v.3.30 builds capability to allow an environment index to be a “survey” of a parameter deviation, we need a larger family of link functions such as logistic and even dome-shaped.

\myparagraph{Specification of Time-Varying Parameters in Long Parameter Lines} 
Selecting time-varying properties for a parameter is specified using element 8 - 14 in the long parameter line setup.  Each element and the options for selection are as described here:

\hypertarget{EnvVar}{}
\begin{itemize}

\item Environmental Link and  Variance (element 8)
	\begin{itemize}
		\item The environmental link and variance (env\_var\&link) matrix is populated with the read environmental data for columns 1-N environmental variables. Alternatively, the environmental link can be linked to derived quantities mapped to columns -1 to -4 where each value corresponds to the following quantities:
		\begin{itemize}
			\item -1;  for ln(relative spawning biomass);
			\item -2;  for recruitment deviation;
			\item -3;  for ln(relative summary biomass) (e.g., current year summary biomass divided by the unfished summary biomass);
			\item -4;  for ln(relative summary numbers).
		\end{itemize}
		\item So, environmental input 103 would use link type 1 (specified in element 9) and apply it to environmental data column 3 and environmental input -103  would use link type 1 and apply it to the "-3" column which is ln(relative summary biomass).
		\item These four derived quantities are all calculated at the beginning of each year within the model, so they are available inside SS to use as the basis for time-varying parameter links without violating any oder of operations rule. 
	\end{itemize}
	
\item Deviation Link (element 9)
	\begin{itemize}
		\item 1 = multiplicative ($P_y*=exp(\text{dev}_y*\text{dev}_{se}$),
		\item 2 = additive ($P_y+=\text{env}_y*\text{dev}_{se}$),
		\item 3 = random walk options are now implemented by using rho in the objective function. SS now expects the estimated deviations to be normal in distribution and the deviation values are multiplied by the standard error parameter as they are used,
		\item 4 = zero reverting random walk with rho. The deviation parameter is now multiplied by the standard error parameter, rather than deviations being penalized according to a specified standard error (the approach in SS v.3.24).
		\item The option of applying the final model year deviation value into the forecast period was added in v. 3.30.13.  This new option is specified by selecting the appropriate deviation link option (1, 2, 3, or 4) and appending a 2 at the front (21, 22, 23, or 24) which will use the final year deviation value for all forecast years. 
	\end{itemize}
	
\item Deviation  Minimum Year (element 10)
	\begin{itemize}
		\item Year for deviations to start for parameter
	\end{itemize}
	
\item Deviation  Maximum Year (element 11)
	\begin{itemize}
		\item Year for deviations to end for parameter
	\end{itemize}
	
\item Deviation Phase (element 12)
	\begin{itemize}
		\item integer, this available element in the long parameter line is now a deviation vector specific phase control
	\end{itemize}
	
\item Blocks (element 13)
	\begin{itemize}
		\item >0: time block index for parameter.
		\item -1: trend with final as offset from base parameter and offset values is in natural log space, also inflection year is in natural log space and the offset from ln(0.5). No additional parameter lines are required.  Three parameters will be estimated; end trend parameter value logistic offset, inflection year logistic offset, and slope.
		\item -2: trend with final as standalone value. No additional parameter lines are required. Three parameters will be estimated; end trend parameter value, inflection year, and slope.
		\item -3 end value is a fraction of base parameter maximum - minimum; inflection year is fraction of end year - start year. No additional parameter lines are required. Three parameters will be estimated; end trend parameter value as a fraction, inflection year as a fraction, and slope.
	\end{itemize}
	
\item Block Functional Form (element 14)
	\begin{itemize}
		\item 0: multiplicative parameter ($P_{y} = P_{base}*exp(tv\_para)$),
		\item 1: additive parameter ($P_{y} = P_{base} + tv\_para$),
		\item 2: replace parameter ($P_{y} = tv\_para$),
		\item 3: random walk across blocks ($P_{block} = P_{block,-1} + tv\_para$),
		\item 4: mean reverting random walk
	\end{itemize}
\end{itemize}

\myparagraph{Block Trends}
Additional information regarding the options for applying blocks (element 13):
\begin{itemize} 
	\item -1: Trend bounded by base parameter minimum maximum and parameters in transformed units (use with caution),
	\begin{itemize}
		\item Logistic approach to trend as offset from base parameter
		\item Transform the base parameter from the parameter section:
		\begin{equation}
			\text{temp} = -0.5*ln\Bigg(\frac{\text{Parm}_{p,LB}-\text{p,Parm}_{UB}+0.0000002}{\text{Parm}_p-\text{Parm}_{p,LB}+0.0000001}-1\Bigg)
		\end{equation}
		\item Add the offset. Note, that offset values in in the transform space.
		%\begin{equation}
		%	\text{temp2} = \sum_{p=1}^{P}\text{TV parameter}{p}MGparm(k+1)
		%\end{equation}
		\item Back transform
		\begin{equation}
			\text{temp1} = \text{Parm}_{p,LB}+\frac{\text{Parm}_{p,UB}-\text{Parm}_{p,LB}}{1+e^{-2*\text{temp}}}
		\end{equation}			
	\end{itemize}
\end{itemize}

%%%%%%%%%%%%%%%%%%%%%%%%%%%%%%%%%%%%%%%%%%%%%%%%%%%%%%%%%%%%%%%%%%%%%%%%%%%%%%%%%%%%%%%%%%%%%%%%%%%%%%%%%%%%%%%%%%%%%
%%%%%%%%%%%%%%%%%%%%%%%%%%%%%%%%%%%%%%%%%%%%%%%%%%%%%%%%%%%%%%%%%%%%%%%%%%%%%%%%%%%%%%%%%%%%%%%%%%%%%%%%%%%%%%%%%%%%%
\hypertarget{2DAR}{}
\subsection{Parameterizing the Two-Dimensional Autoregressive Selectivity}
When the two-dimensional autoregressive selectivity feature is turned on for a fleet, the selectivity is calculated as a product of the assumed selectivity pattern and a non-parametric deviation term deviating from this assumed pattern:

\begin{equation}
\hat{S}_{a,t} = S_aexp^{\epsilon_{a,t}}
\end{equation}

where $S_a$ is specified in the corresponding age/length selectivity types section and it can be either parametric (recommended) or non-parametric (including any of the existing selectivity options in SS); $\epsilon_{a,t}$ is simulated as a two-dimensional first-order autoregressive (2D AR1) process:

\begin{equation}
vec(\epsilon) \sim MVN(\mathbf{0},\sigma_s^2\mathbf{R_{total}})
\end{equation}

where $\epsilon$ is the two-dimensional deviation matrix and $\sigma_s^2\mathbf{R_{total}}$ is the covariance matrix for the 2D AR1 process. More specifically, $\sigma_s^2$ quantifies the variance in selectivity deviations and $\mathbf{R_{total}}$ is equal to the kronecker product ($\otimes$) of the two correlation matrices for the among-age and among-year AR1 processes:

\begin{equation}
\mathbf{R_{total}}=\mathbf{R}\otimes\mathbf{\tilde{R}}
\end{equation}

\begin{equation}
\mathbf{R}_{a,\tilde{a}}=\rho_a^{|a-\tilde{a}|}
\end{equation}

\begin{equation}
\mathbf{\tilde{R}}_{t,\tilde{t}}=\rho_t^{|t-\tilde{t}|}
\end{equation}

where $\rho_a$ and $\rho_t$ are the among age and among year AR1 coefficients, respectively. When both of them are zero, $\mathbf{R}$ and $\mathbf{\tilde{R}}$ are two identity matrices and their Kronecker product, $\mathbf{R_{total}}$, is also an identity matrix. In this case selectivity deviations are essentially identical and mutually independent:

\begin{equation}
\epsilon_{a,t}\sim N(0,\sigma_s^2)
\end{equation} 

\myparagraph{Using the Two-Dimensional Autoregressive Selectivity}
Note, \citet{xu_new_2019} has additional information on tuning the 2D AR selectivity parameters. First, fix the two AR1 coefficients ($\rho_a$ and $\rho_t$) at 0 and tune $\sigma_s$ iteratively to match the relationship:

\begin{equation}
\sigma_s^2=SD(\epsilon)^2+\frac{1}{(a_{max}-a_{min}+1)(t_{max}-t_{min}+1)}\sum_{a=a_{min}}^{a_{max}}\sum_{t=t_{min}}^{t_{max}}SE(\epsilon_{a,t})^2
\end{equation}

The minimal and maximal ages/lengths and years for the 2D AR1 process can be freely specified by users in the control file. However, we recommend specifying the minimal and maximal ages and years to cover the relatively "data-rich" age/length and year ranges only. Particularly we introduce: 

\begin{equation}
b=1-\frac{\frac{1}{(a_{max}-a_{min}+1)(t_{max}-t{min}+1)}\sum_{a=a_{min}}^{a_{max}}\sum_{t=t_{min}}^{t_{max}}SE(\epsilon_{a,t})^2}{\sigma_s^2}
\end{equation}

as a measure of how rich the composition data is regarding estimating selectivity deviations. We also recommend using the Dirichlet-Multinomial method to "weight" the corresponding composition data while $\sigma_s$ is interactively tuned in this step.

Second, fix $\sigma_s$ at the value iteratively tuned in the previous step and estimate $\epsilon_{a,t}$. Plot both Pearson residuals and $\epsilon_{a,t}$ out on the age-year surface to check their 2D dimensions. If their distributions seems to be not random but rather be autocorrelated (deviation estimates have the same sign several ages and/or years in a row), users should consider estimating and then including the autocorrelations in $\epsilon_{a,t}$.

Third, extract the estimated selectivity deviation samples from the previous step for estimating $\rho_a$ and $\rho_t$ externally by fitting the samples to a stand-alone model written in Template-Model Builder (TMB). In this model, both $\rho_a$ and $\rho_t$ are bounded between 0 and 1 via applying a logic transformation. If at least one of the two AR1 coefficients are notably different from 0, SS should be run one more time by fixing the two AR1 coefficients at their values externally estimated from deviation samples. The Pearson residuals and $\epsilon_{a,t}$ from this run are expected to distribute more randomly as the  autocorrelations in selectivity deviations can be at least partially included in the 2D AR1 process.


%%%%%%%%%%%%%%%%%%%%%%%%%%%%%%%%%%%%%%%%%%%%%%%%%%%%%%%%%%%%%%%%%%%%%%%%%%%%%%%%%%%%%%%%%%%%%%%%%%%%%%%%%%%%%%%%%%%%%
%%%%%%%%%%%%%%%%%%%%%%%%%%%%%%%%%%%%%%%%%%%%%%%%%%%%%%%%%%%%%%%%%%%%%%%%%%%%%%%%%%%%%%%%%%%%%%%%%%%%%%%%%%%%%%%%%%%%%

\subsection{Continuous seasonal recruitment}
It is awkward in SS to set up a seasonal model such that recruitment can occur with similar and independent probability in any season of any year.  Consequently, some users have attempted to setup SS so that each quarter appears as a year.  They have set up all the data and parameters to treat quarters as if they were years (i.e., each still has a duration of 1.0 time step).  This can work, but requires that all rate parameters be re-scaled to be correct for the quarters being treated as years.

Another option is available.  If there is one season per year and the season duration is set to 3 (rather than the normal 12), then the season duration is calculated to be 3/12 or 0.25. This means that the rate parameters can stay in their normal per year scaling and this shorter season duration makes the necessary adjustments internally. Some other adjustments to make when doing quarters as years include:

\begin{itemize}
	\item Re-index all "year seas" inputs to be in terms of quarter-year because all are now season 1; increase end year (endyr) value in sync with this.
	\item Increase max age because age is now in quarters.
	\item In the age error definitions, increase the number of entries in accord with new max age
	\item In the age error definitions, recode so that each quarter-age gets assigned to the correct age bin. This is because the age data are still in terms of age bins; i.e., the first 4 entries for quarter-ages 1 through 4 will all be assigned to age bin 1.5; the next four to age bin 2.5;  you cannot accomplish the same result by editing the age bin values because the standard deviation of ageing error is in terms of age bin.
	\item In the control file, multiple the natural mortality age breakpoints and growth Amin and Amax values by 1/season duration.
	\item Decrease the R0 parameter starting value because it is now the average number of recruitments per quarter year.
	\item Edit the recruitment deviation (rec\_dev) start and end years to be in terms of quarter year.
	\item Edit any age selectivity parameters that refer to age to now refer to quarter age.
	\item If there needs to be some degree of seasonality to recruitment or some parameter, then you could create a cyclic pattern in the environmental input and make recruitment or some other parameter a function of this cyclic pattern.
\end{itemize}

A good test showing comparability of the 3 approaches to setting up a quarterly model should be done.

\pagebreak

%%%%%%%%%%%%%%%%%%%%%%%%%%%%%%%%%%%%%%%%%%%%%%%%%%%%%%%%%%%%%%%%%%%%%%%%%%%%%%%%%%%%%%%%%%%%%%%%%%%%%%%%%%%%%%%%%%%%%
%%%%%%%%%%%%%%%%%%%%%%%%%%%%%%%%%%%%%%%%%%%%%%%%%%%%%%%%%%%%%%%%%%%%%%%%%%%%%%%%%%%%%%%%%%%%%%%%%%%%%%%%%%%%%%%%%%%%%

\section{Detailed Information on SS Processes}

The processes and calculations with SS can be complex and not transparent based on the model input files. Here additional information is provided to users to assist in understanding some of these processes.

\subsection{Jitter}
\hypertarget{Jitter}{}
The jitter function has been updated with SS v.3.30.  The following steps are now performed to determine the jittered starting parameter values (illustrated in Figure \ref{fig:jitter}):
\begin{enumerate}
	\item A normal distribution is calculated such that the pr(P\textsubscript{MIN}) = 0.1\% and the pr(P\textsubscript{MAX}) = 99.9\%.
	\item A jitter shift value, termed "\textit{K}", is calculated from the distribution equal to pr(P\textsubscript{CURRENT}).
	\item A random value is drawn, "\textit{J}", from the range of \textit{K}-jitter to \textit{K}+jitter.
	\item Any value which falls outside the 0-1 range (in the cumulative normal space) is mapped back from the bound to a point one-tenth of the way from the bound to the initial value.
	\item \textit{J} is a new cumulative normal probability value.
	\item Calculate a new parameter value, P\textsubscript{JITTERED}, such that pr(P\textsubscript{JITTERED}) = \textit{J}.
\end{enumerate}

\begin{figure}[h]
	\begin{center}
		\includegraphics[scale = 0.75]{jitter_illustration}\\
		\caption{Illustration of the jitter algorithm}
		\label{fig:jitter}
	\end{center}
\end{figure}

In SS, the jitter fraction defines a uniform distribution in cumulative normal space +/- the jitter fraction from the initial value (in cumulative normal space). The normal distribution for each parameter, for this purpose, is defined such that the minimum bound is at 0.001, and the maximum at 0.999 of the cumulative distribution. If the jitter faction and original initial value are such that a portion of the uniform distribution goes beyond 0.001 or 0.999 of the cumulative normal, the new value is set to one-tenth of the way from the bound to the original initial value. 

Therefore sigma = (max-min) / 6.18. For parameters that are on the log-scale, sigma may be the correct measure of variation for jitters, for real-space parameters, CV (= sigma/original initial value) may be a better measure. 

If the original initial value is at or near the middle of the min-max range, then for each 0.1 of jitter, the range of jitters extends about 0.25 sigmas to either side of the original value (though as the total jitter increases the relationship varies more than this), and the average absolute jitter is about half of that.  For values far from the middle of the min-max range, the resulting jitter is skewed in parameter space, and may hit the bound, invoking the resetting mentioned above. 

To evaluate the jittering, the bounds, and the original initial values, a jitter\_info table is available from r4ss, including sigma, CV and InitLocation columns (the latter referring to location within the cumulative normal – too close to 0 or 1 indicates a potential issue).

Note: parameters with min $\leq$ -99 or max $\geq$ 999 are not jittered to avoid unreasonable values (a warning is produced to indicate this).

\hypertarget{PriorDescrip}{}
\subsection{Parameter Priors}
Priors on parameters fulfill two roles in SS.  First, for parameters provided with an informative prior, SS is receiving additional information about the true value of the parameter.  This information works with the information in the data through the overall log likelihood function to arrive at the final parameter estimate.  Second, diffuse priors provide only weak information about the value of a prior and serve to manage model performance during execution.  For example, some selectivity parameters may become unimportant depending upon the values of other parameters of that selectivity function.  In the double normal selectivity function, the parameters controlling the width of the peak and the slope of the descending side become redundant if the parameter controlling the final selectivity moves to a value indicating asymptotic selectivity.  The width and slope parameters would no longer have any effect on the log likelihood, so they would have no gradient in the log likelihood and would drift aimlessly.  A diffuse prior would then steer them towards a central value and avoid them crashing into the bounds.  Another benefit of diffuse priors is the control of parameters that are given unnaturally wide bounds.  When a parameter is given too broad of a bound, then early in a model run it could drift into this tail and potentially get into a situation where the gradient with respect that parameter approaches zero even though it is not at its global best value.  Here the diffuse prior helps move the parameter back towards the middle of its range where it presumably will be more influential and estimable.  

The options for parameter priors are described as a function of $Pval$, the value of the parameter for which a prior is being calculated, as well as the parameter bounds in the case of the beta distribution ($Pmax$ and $Pmin$), and the input values for $Prior$ and $Pr\_SD$, which in some cases are the mean and standard deviation, but interpretation depends on the prior type. The Prior Likelihoods below represent the negative log likelihood in all cases.

\myparagraph{Prior Types}
Note that the numbering in SS v.3.30 is different from that used in SS v.3.24 (where confusingly -1 indicated no prior and 0 indicated a normal prior). The calculation of the negative log likelihood is provided below for each prior types, as a function of the following inputs:

\begin{tabular}{ll}
	$P_\text{init}$ & The value of the parameter for which a prior is being calculated where init can either be\\
	                & the initial un-estimated value or the estimated value (3rd column in control or \\
	                & control.ss\_new file)       \\
	$P_\text{LB}$   & The lower bound of the parameter (1st column in control file)     \\
	$P_\text{UB}$   & The upper bound of the parameter (2nd column in control file)     \\
	$P_\text{PR}$   & The input value for the prior input (4th column in control file)  \\
	$P_\text{PRSD}$ & The standard deviation input value for the prior (5th column in control file) \\
\end{tabular}

\begin{itemize}
	\item  \textbf{Prior Type = 0 = No prior applied} \\ 
	In a Bayesian context this is equivalent to a uniform prior between the parameter bounds.
	
	\item  \textbf{Prior Type = 1 = Symmetric beta prior} \\ 
	The symmetric beta is scaled between parameter bounds, imposing a larger penalty near the bounds.  Prior standard deviation of 0.05 is very diffuse and a value of 5.0 provides a smooth U-shaped prior. The prior input is ignored for this prior type.
	\begin{equation} 
		\mu = -P_\text{PRSD} \cdot ln\left(\frac{P_\text{UB}+P_\text{LB}}{2} - P_\text{LB} \right) - P_\text{PRSD} \cdot ln(0.5)
	\end{equation}
	
	\begin{equation}
		\begin{split}
			\text{Prior Likelihood} = & -\mu -P_\text{PRSD} \cdot ln\left(P_\text{init}-P_\text{LB}+0.0001\right) \\
			& - P_\text{PRSD} \cdot ln\left(1-\frac{P_\text{init}-P_\text{LB}-0.0001}{P_\text{UB}-P_\text{LB}}\right)
		\end{split}
	\end{equation}

	\begin{figure}[h]
	\begin{center}
		\includegraphics[scale = 0.6]{SymetricBeta}\\
	\end{center}
	\caption{The shape of the symmetric beta prior across alternative standard deviation values and the change in the negative log-likelihood.}
	\end{figure}	

	
	\item \textbf{Prior Type = 2 = Beta prior}  \\ 
	The definition of $\mu$ is consistent with CASAL's formulation with the $\beta_\text{PR}$ and $\alpha_\text{PR}$ corresponding to the $m$ and $n$ parameters.
	\begin{equation}
		\mu = \frac{P_\text{PR}-P_\text{LB}}{P_\text{UB}-P_\text{LB}} 
	\end{equation}
	\begin{equation}
		\tau  = \frac{(P_\text{PR}-P_\text{LB})(P_\text{UB}-P_\text{PR})}{P_\text{PRSD}^2}-1
	\end{equation}
	\begin{equation}
		\beta_\text{PR}  = \tau \cdot \mu
	\end{equation}
	\begin{equation}
		\alpha_\text{PR} = \tau (1-\mu)
	\end{equation}
	
	\begin{equation}
		\begin{split}
			\text{Prior Likelihood} = & (1 - \beta_\text{PR}) \cdot ln(0.0001 + P_\text{init} - P_\text{LB}) \\
			& + (1 - \alpha_\text{PR}) \cdot ln(0.0001 + P_\text{UB} - P_\text{init}) \\
			& - (1 - \beta_\text{PR}) \cdot ln(0.0001 + P_\text{PR} - P_\text{LB}) \\
			& - (1 - \alpha_\text{PR}) \cdot ln(0.0001 + P_\text{UB} - P_\text{PR})
		\end{split}
	\end{equation}

	%\begin{figure}[h]
	%\begin{center}
	%	\includegraphics[scale = 0.9]{BetaComparison}\\
	%\end{center}
	%\caption{Comparison of the symmetric beta and the beta prior functions.}
	%\end{figure}	

	
	\item \textbf{Prior Type 3 = Lognormal prior} \\ 
	Note that this is undefined for $p <= 0$ so the lower bound on the parameter must be > 0. The prior value is input into the parameter line in natural log space while the initial parameter value is defined in normal space (e.g., init = 0.20, prior = -1.609438).
	\begin{equation}
		\text{Prior Likelihood} = \frac{1}{2} \left(\frac{ln(P_\text{init})-P_\text{PR}}{P_\text{PRSD}}\right)^2
	\end{equation}
	
	\item \textbf{Prior Type 4 = Lognormal prior with bias correction} \\ 
	This option allows the prior mean value to be entered as the ln(mean). Note that this is undefined for $p <= 0$ so the lower bound on the parameter must be > 0.
	\begin{equation}
		\text{Prior Likelihood} = \frac{1}{2} \left(\frac{ln(P_\text{init})-P_\text{PR} + \frac{1}{2}{P_\text{PRSD}}^2}{P_\text{PRSD}}\right)^2
	\end{equation}
	
	\item \textbf{Prior Type 5 = Gamma prior} \\ 
	The lower bound should be 0 or greater.
	\begin{equation}
		\text{scale} = \frac{{P_\text{PRSD}}^2}{P_\text{PR}}
	\end{equation}
	\begin{equation}
		\text{shape} = \frac{P_\text{PR}}{\text{scale}}
	\end{equation}
	\begin{equation}
		\text{Prior Likelihood} = -\text{shape} \cdot ln(\text{scale}) - ln\big(\Gamma(\text{shape})\big) + (\text{shape} - 1) \cdot ln(P_\text{init}) - \frac{P_\text{init}}{\text{scale}}
	\end{equation}
	
	\item \textbf{Prior Type 6 = Normal prior} \\ 
	Note that this function is independent of the parameter bounds.
	\begin{equation}
		\text{Prior Likelihood} = \frac{1}{2} \left(\frac{P_\text{init} - P_\text{PR}}{P_\text{PRSD}}\right)^2
	\end{equation}
\end{itemize}

%=========Forecast Module
\input{_forecast_module}

%=========F mortality in SS
\subsection{Fishing Mortality in Stock Synthesis}

The implementation and reporting of fishing mortality rate, $F$, in SS3 has some aspects that can be confusing.  This description provides an overview of the ways in which $F$ is calculated, used, and reported.  

\myparagraph{Rationale}
Fishery management systems expect to have a measure of annual fishing mortality that describes the intensity of the fishery such that an optimal level of $F$ can be articulated and accountability measures can be invoked if $F$ is too high, e.g., overfishing.  This concept is simple and straightforward if the model is a simple biomass dynamics such that a single annual $F$ value operates on the entirety of a non-age structured population. It also is simple for age-structured models that have a single fishing fleet and knife-edge selectivity beginning at some specified age.

The simplicity of $F$ disappears quickly as models invoke a variety of realistic complexities such as: allowing the $F$ to differ among ages or to be based on size; using a collection of fleets with different $F$ levels and different age patterns for $F$; spreading the population across areas and allowing different fleets with different $F$ among the areas.  An unambiguous measure of annual fishing intensity that represents the cumulative effect of all that complexity has not been defined.  This problem has not been solved with SS3, but some logical alternatives have been made available.

\myparagraph{Nomenclature}
The nomenclature below ignores sex, morphs and areas for simplicity. The quantities associated with $F$ calculations are defined as:

$f$ is fleet.

$t$ is a time step; continuous across years $y$ and seasons $s$; equivalent to year if only 1 season.

$a$ is age.

$C_{t,f}$ is fleet-specific catch in a time step.

$B_{t,f}$ is fleet specific available biomass, e.g., total biomass filtered by fleet-specific age selectivity, $s_{t,f,a}$.

$s_{t,f,a}$ is age-specific selectivity for a fleet. If selectivity is length-specific, then age-specific selectivity is calculated as the dot product across length bins of length selectivity and the normal (or lognormal) distribution of length-at-age.  If selectivity is both length- and age-based, which is an entirely normal concept in SS3, then age selectivity due to length selectivity is calculated first, then multiplied by the direct age selectivity.  This compound age selectivity is used in the mortality calculations and is reported as asel2 in report file.  See appendix to \citet{methotstock2013} for more detail on this.

$F_{t,f}'$ is the apical fishing mortality for a fleet. This means that it is the rate for the age that has selectivity equal to 1.0. If your model is using $F'$s as parameters, then the parameter values are for $F'$.

$F_{t,f,a}$ is age and fleet-specific fishing mortality rate equal to $F_{t,f}' * s_{t,f,a}$. Note that it is possible for no age to have a selectivity equal to 1.0. In this case, $F'$ is still the rate for the hypothetical age that has selectivity equal to 1.0. The reported $F'$ values are not rescaled to be an $F$ for the age with peak selectivity. Users need to take this into account if they are comparing reported $F'$ values to reported vector of $F_{t,f,a}$ values.

$\text{ann}F_y$ is a measure of the total fishing intensity for a year, based on one of several user-specified options (see below).

$F\text{std}_y$ is a standardized measure of the total fishing intensity for a year and is reported in the derived quantities, so variance is calculated for this quantity. See below for how it relates to $annF$.

Terminology and reporting of $\text{ann}F$ and $F\text{std}$ has been slightly revised for clarity in v.3.30.15.00 and the description here follows the new conventions.

\myparagraph{$F$ Calculation}
SS3 allows for three approaches to estimate the $F'$ that will match the input values for retained catch. Note that SS3 is calculating the $F'$ to match the retained catch conditional on the fraction of total catch that is retained, e.g., the total catch is partitioned into retained and discarded portions.

\begin{enumerate}
	\item Pope's method decays the numbers-at-age to the middle of the season, calculates a harvest rate for each fleet, $H_{t,f}$, that is the ratio of $C_{t,f}$ to $B_{t,f}$, then decays the survivors to the end of the season. The total mortality, $Z_{t,a}$, from the ratio of survivors to initial numbers, is then calculated. The $Z$ is subsequently used for in-season interpolation to get expected values for observations.
	
	\item $F$ as parameters uses the standard Baranov catch equation and lets ADMB find the $F'$ parameter values that produce the lowest negative log-likelihood, which includes fit to the input catch data. $F$ as parameters method tends to work better than Pope's or hybrid in high $F$ situations because it allows for some lack of fit to catch levels in early iterations and can later improve this fit as it closes in on the best solution.
	
	\item Hybrid $F$ starts by calculating a harvest rate, $H$, using Pope's, then converts these $H$ values, which have units of fractional harvest rate, into an approximate of $F'$ in exponential units, tuning these $F'$ values over a few iterations to get a better match to each fleet's catch.
\end{enumerate}

Items to note:
\begin{itemize}
	\item SS3 includes a permutation on the $F$ as parameters method. In the first few phases, SS3 uses hybrid, then between phases it converts these directly calculated $F'$ values into parameters and proceeds in subsequent phases and MCMC to use the parameter approach. This variation on the parameter method is the recommend approach in high $F$ situations.
	
	\item With Pope's method, the $H$ values are fraction caught, so duration of the season does not matter. Parameter and hybrid treat $F'$ identically and multiply the $F'$ values by season duration (which has units of fraction of a year) as it is used. Each of the $F$ methods ends up with a $Z_{t,f}$ that is used for in-season interpolation.
\end{itemize}

\myparagraph{Relative $F$ and $F$mult}
The $F'$ is fleet-specific, so it is useful to have a concept of relative $F$, $\text{rel}F_f$, among fleets. In SS3, $\text{rel}F_f= F_{t,f}'/\sum_{f}^{}F_{t,f}'$ for a single time period $t$. In the benchmark and forecast routines, SS3 can calculate $\text{rel}F_f$ using $F_{t,f}'$ over a range of years, or the user can input custom $\text{rel}F$ values for benchmark and forecast in the forecast.ss file. Note that in a multi-season model setup, $\text{rel}F_f$ is implemented as $\text{rel}F_{s,f}$ where $s$ is the season. These get multiplied by season duration as they are used.

In the benchmark section of the code, SS3 searches for an $F$mult to achieve various management reference points (often referred to as benchmarks). In this search, SS3 calculates a benchmark $F$ as  $F_{ben,f}' = F\text{mult} * \text{rel}F_f$, then calculates equilibrium yield and spawning biomass per recruit (SPR). SS3 searches for the $F$mult that satisfies the search conditions, first for user-specified SPR, then for user-specified spawning biomass at a management target (B\textsubscript{TGT} or $F_{0.1}$), then for MSY. The resultant benchmark quantities are reported in the derived quantities, but $F$mult and $F_{ben,f}'$ are only reported in the Forecast\_report.sso file. SS3 stores the benchmark $F$mult values so that user can invoke them for the forecast.

\myparagraph{Annual $F$}
The $\text{ann}F$ is a single annual value across all fleets and areas according to F\_report\_units, which is specified by users in the starter file. If there are many fleets, across several areas and with very different selectivity patterns, $\text{ann}F$ can have a complicated relationship to apical $F$. The F\_report\_units specification in the starter.ss file, see example line below, allows user to calculate it using $F'$ directly, use exploitation rate, or be derived from $Z$-at-age.

Example $F$ reporting unit specification in the starter.ss file:

\begin{center}
	\begin{longtable}{p{2cm} p{12cm}}
		\hline
		5 & \# F\_report\_units:\Tstrut\\
		  & 0 = skip; \\
		  & 1 = exploitation(Bio); \\
		  & 2 = exploitation(Num); \\ 
		  & 3 = sum(Frates); \\
		  & 4 = true F for range of ages; \\
		  & 5 = unweighted avg. F for range of ages. \Bstrut\\
		\hline
		3 7 & \# min and max age over which average F will be calculated \Tstrut\Bstrut\\
		\hline
	\end{longtable}
\end{center}

For options 4 and 5 of F\_report\_units, the $F$ is calculated as $Z-M$ where $Z$ is calculated as $ln(N_{t+1,a+1}/N_{t,a})$, thus $Z$ subsumes the effect of $F$.

The ann$F$ is calculated for each year of the estimated time series and of the forecast. Additionally, an ann$F$ is calculated in the benchmark calculations to provide equilibrium values that have the same units as ann$F$ from the time series. In versions previous to v.3.30.15, it was labeled inaccurately as $F$std in the output, not ann$F$. For example, in the Management Quantities section of derived quantities prior to v.3.30.15, there is a quantity labeled Fstd\_Btgt. This is more accurately labeled as the annual $F$ associated with the biomass target, ann\_F\_Btgt, in v.3.30.15.

\myparagraph{$F$std}
$F$std is a single annual value based on ann$F$ and the relationship to ann$F$ is specified by F\_report\_basis in the starter.ss file. The benchmark ann$F$ may be used to rescale the time series of ann$F$s to become a time series of standardized values representing the intensity of fishing, $F$std. The report basis is selected in the starter file as:

\begin{center}
	\begin{longtable}{p{2cm} p{12cm}}
		%\multicolumn{2}{l}{The starter file line:}\\
		\hline
		0 & \# F\_report\_basis: \Tstrut\\
		& 0 = raw F report; \\
		& 1 = F / F\textsubscript{SPR}; \\ 
		& 2 = F / F\textsubscript{MSY}; \\
		& 3 = F / F\textsubscript{BTGT}.\Bstrut\\
		\hline
	\end{longtable}
\end{center}

For example, if user selects option 1, $F$ / $F_\text{SPR}$, the time series of ann$F$ will be divided by each value by the ann$F$ calculated in benchmark.

\myparagraph{Units for Stock Synthesis inputs related to $F$}
Below is a list of items to consider in terms of units for $F$ in SS3:
\begin{itemize}
	\item If F\_ballpark is specified in the control.ss file, its units are the same as ann$F$, so is not fleet-specific.
	
	\item $F$ as parameter values has units of fleet-specific apical $F'$.
	
	\item In the forecast.ss file there is an option to input a vector of rel$F$ values. These are dimensionless and will be rescaled to sum to 1.0.
	
	\item In the forecast.ss file there is an option to specify an $F$ scalar for the forecast.  The units of $F$ scalar are the same as the $F$mult values calculated in benchmark.  There are a full set of options for forecast $F$ scalar that can be selected in the forecast file 
	%(-1 = none; 0 = simple; 1 = F\textsubscript{SPR}; 2 = F\textsubscript{MSY} 3 = F\textsubscript{BTGT} or F\textsubscript{0.1}; 4 = Ave F (uses first-last relative F years); and 5 = input annual F scalar). 
	If the forecast $F$ scalar is set as $F_\text{SPR}$, then SS3 will use SPR\_Fmult calculated in benchmark and reported in Forecast-report.sso.  If user selects the option to input an annual $F$ scalar, option 5, then the value is input on a following line.  Whichever method the user selects for forecast $F$ scalar ($F$mult), SS3 will start the forecast by creating a fleet-specific vector of apical $F$ values from $F$mult*rel$F_f$.
	
	\item Also in the forecast.ss file, the last section of inputs allows for input of time and fleet specific apical $F_{t,f}'$ values that override the basic forecast $F$ specification described above.
\end{itemize}


\pagebreak