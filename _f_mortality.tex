\subsection{Fishing Mortality in Stock Synthesis}

The implementation and reporting of fishing mortality rate, $F$, in SS3 has some aspects that are more complex than in simpler models. This description provides an overview of the ways in which $F$ is calculated, used, and reported.  

\myparagraph{Rationale}
Fishery management systems expect to have a measure of annual fishing mortality ($F$) that describes the intensity of the fishery such that an optimal level of $F$ can be articulated and accountability measures can be invoked if $F$ is too high, e.g., overfishing. This concept is simple and straightforward if the model is a simple biomass dynamics model such that a single annual $F$ value operates on the entirety of a non-age structured population. It also is simple for age-structured models that have a single fishing fleet and knife-edge selectivity beginning at some specified age. The simplicity of $F$ disappears quickly as models invoke a variety of realistic complexities such as: allowing the $F$ to differ among ages or to be based on size; using a collection of fleets with different $F$ levels and different age patterns for $F$; spreading the population across areas and allowing different fleets with different $F$ among the areas. An unambiguous measure of annual fishing intensity that represents the cumulative effect of all that complexity is a challenge. This problem has not been solved with SS3, but some logical alternatives have been made available.

\myparagraph{Nomenclature}

The quantities associated with $F$ calculations are defined as:

$f$ is fleet.

$t$ is a time step; continuous across years $y$ and seasons $s$; equivalent to year if only 1 season.

$a$ is age.

$C_{t,f}$ is fleet-specific catch in a time step.

$s_{t,f,a}$ is age-specific selectivity for a fleet. If selectivity is length-specific, then age-specific selectivity due to length-selectivity is calculated as the dot product across length bins of length selectivity and the normal (or lognormal) distribution of length-at-age. If selectivity is both length- and age-based, which is an entirely normal concept in SS3, then age selectivity due to length selectivity is calculated first, then multiplied by the direct age selectivity. This compound age selectivity is used in the mortality calculations and is reported as $Asel2$ in report:32 of report.sso. See appendix to \citet{methotstock2013} for more detail on this.

Selectivity can be sex-specific, and different growth morphs and platoons can have different age-selectivity due to the effect of length-selectivity on their unique size-at-age. This added dimension, $g$, for biological group is not included in the nomenclature here but exists in all the SS3 calculations.

$B_{t,f}$ is fleet specific available biomass, e.g., total biomass filtered by fleet-specific age selectivity, $s_{t,f,a}$. Note that this is not adjusted by the $max(s_{t,f,a})$.

$F_{t,f}'$ is a fleet's fishing mortality for the age that has selectivity equal to 1.0. This is termed F\_scalar or  $F'$ in the SS3 system. If your model is using $F'$s as parameters, then the parameter values are for the $F'$. Note that some selectivity curves, like double normal, are explicit about having a maximum of 1.0. But other curves like logistic and combinations of length-selectivity and growth, may produce an age-selectivity curve that never reaches 1.0 and some situations, especially 2DAR, will produce selectivity >1.0 routinely. In all cases, the resultant $F_{t,f,a}$ comes from $F_{t,f}' * s_{t,f,a}$. $F_{t,f,a}$ is output in report:32 with the rows labelled simply as "F". The reported $F'$ values are never rescaled to be the $F$ for the age with peak selectivity. Users need to take this into account if they are comparing reported $F'$ values to reported vector of true $F_{t,f,a}$ values.

Apical selectivity is the maximum age-specific selectivity and is not explicit in any internal calculation in SS3, it is just for reporting. If selectivity has a maximum value of 1.0, then apical\_F and F\_scalar are identical. You can find apical selectivity reported as maximum\_ASEL2, immediately after report:32.

\myparagraph{$F$ Calculation}
SS3 allows for three approaches to estimate the $F'$ that will match the input values for retained catch. Note that SS3 is calculating the $F'$ to match the retained catch conditional on the fraction of total catch that is retained, e.g., the total catch can be partitioned into retained and discarded portions.

\begin{enumerate}
	\item Pope's method decays the numbers-at-age to the middle of the season, calculates a harvest rate for each fleet, $H_{t,f}$, that is the ratio of $C_{t,f}$ to $B_{t,f}$, then decays the survivors to the end of the season. The total mortality, $Z_{t,a}$, from the ratio of survivors to initial numbers, is then calculated. The $Z$ is subsequently used for in-season interpolation to get expected values for observations.
	
	\item $F$ as parameters method uses the standard Baranov catch equation and lets ADMB find the $F'$ parameter values that produce the lowest negative log-likelihood, which includes fit to the input catch data. $F$ as parameters method tends to work better than Pope's method or hybrid $F$ method in high $F$ situations because it allows for some lack of fit to catch levels in early iterations and can later improve this fit as it closes in on the best solution.
	
	\item Hybrid $F$ method starts by calculating a harvest rate, $H$, using Pope's method, then converts these $H$ values, which have units of fractional harvest rate, into an approximate of $F'$ in exponential units, tuning these $F'$ values over a few iterations to get a better match to each fleet's catch.
\end{enumerate}

Items to note:
\begin{itemize}
	\item SS3 includes a permutation on the $F$ as parameters method. In the first few phases, SS3 uses the hybrid $F$ method, then between phases it converts these directly calculated $F'$ values into parameters and proceeds in subsequent phases and MCMC to use the parameter approach. This variation on the parameter method is the recommend approach in high $F$ situations.
	
	\item With Pope's method, the $H$ values are fraction caught, so duration of the season does not matter. The $F$ as parameters and hybrid $F$ method treat $F'$ identically and multiply the $F'$ values by season duration (which has units of fraction of a year) as it is used. Each of the $F$ methods ends up with a $Z_{t,f}$ that is used for in-season interpolation.
\end{itemize}

\myparagraph{Annual\_F and F\_std reporting}
SS3 provides several options for reporting overall annual fishing intensity. $F\text{std}_y$ is an ADMB-specific derived quantities, so its variance is calculated. Annual\_F (annF) is a building block for F\_std and provides additional reporting of some of the same quantities.

The options for F\_std reporting are in the starter.ss file:
\begin{center}
	\begin{longtable}{p{2cm} p{12cm}}
		\hline
		5 & \# F\_std\_reporting\_units: \Tstrut\\
		  & 0 = skip; \\
		  & 1 = exploitation(Bio); \\
		  & 2 = exploitation(Num); \\ 
		  & 3 = sum($F$ scalars*seas); \\
		  & 4 = mean $F$ for range of ages (numbers weighted); \\
		  & 5 = unweighted mean $F$ for range of ages. \Bstrut\\
		\hline
		3 7 & \# min and max age over which mean $F$ will be calculated with F\_reporting = 4 or 5 \Tstrut\Bstrut\\
		\hline
	\end{longtable}
	\vspace*{-1.7\baselineskip}
\end{center}

\begin{itemize}
	\item For options 1 and 2, the numerator is retained catch numbers or biomass, and the denominator is summary numbers or summary biomass.
	\item For option 3, the result is simply the season weighted sum of the $F$'s (except no season weighting if using Pope's harvest rate approach). Option 3 will be misleading in a multi-area model as the calculation is insensitive to the proportion of the population in the area where the F is being applied.
	\item For options 4 and 5, the $F$ is calculated as $Z-M$ where $Z$ is calculated as $ln(N_{t+1,a+1}/N_{t,a})$, thus $Z$ subsumes the effect of $F$. In a multi-area model, the N values are summed across areas so counters the shortcoming of option 3, but the reported value is buffered if there in a large portion of the population in a lightly fished area.
	\item The F\_std is calculated for each year of the estimated time series and of the forecast. Additionally, a value with the same units is calculated in the benchmark calculations to provide a basis for scaling the output. These benchmark values are reported in the Mgmt\_Quantity section of derived quantities with labels like annF\_Btgt. 
	\item Prior to v.3.30.15, these quantities were inaccurately labeled Fstd\_Btgt.
\end{itemize}

F\_std scaling is selected in the starter file as:
\begin{center}
	\begin{longtable}{p{2cm} p{12cm}}
		%\multicolumn{2}{l}{The starter file line:}\\
		\hline
		0 & \# F\_std\_scaling: \Tstrut\\
		& 0 = no scaling; \\
		& 1 = $F$ / $F_{SPR}$; \\ 
		& 2 = $F$ / $F_{MSY}$; \\
		& 3 = $F$ / $F_{BTGT}$.\Bstrut\\
		\hline
	\end{longtable}
	\vspace*{-1.7\baselineskip}
\end{center}

Note that $F$ means annual F\_std, $F_{MSY}$ means F\_std at MSY.

The results of these calculations is displayed most explicitly in the report.sso table "EXPLOITATION report:14". Here, the table columns are:

Yr Seas Seas\_dur F\_std annual\_F annual\_M <each fleet's F\_scalar>

In this table, the displayed value for annual\_F will be from the $F=Z-M$ method regardless of which option was chosen for F\_std.  If F\_std uses option 4 or 5, then the annual\_F will use the same range of ages. Otherwise, annual\_F will be for the age that is the mid-age of the age range.

% While F is a simple concept in a biomass dynamics model with a single fishing fleet, the concept of "F" as a single number is very incomplete when there are multiple fleets, some with length-based and/or dome-shaped selectivity. In SS3, the F multiplier is multiplied first by fleet-specific relative F's to get the F for each fleet, then that fleet-specific F is multiplied by the age-selectivity to get F at age. Depending on the choice of selectivity pattern, these age-specific F's may or may not peak at 1.0 and can even exceed 1.0 in some circumstances. The realized F-at-age for each fleet is (F multiplier) * (relative F) * (age-selectivity). You can see the results in the F\_AT\_AGE section of the Report file. 

% While F multiplier is always done as just described, the user has various options for the reporting of the realized F. These are in starter.ss and described in the manual.

\myparagraph{Relative $F$ and Fmult}
The $F'$ is fleet-specific, so it is useful to have a concept of relative $F$, $\text{rel}F_f$, among fleets. In SS3, $\text{rel}F_f= F_{t,f}'/\sum_{f}^{}F_{t,f}'$ for a single time period $t$. In the benchmark and forecast routines, SS3 can calculate $\text{rel}F_f$ using $F_{t,f}'$ over a range of years, or the user can input custom $\text{rel}F$ values for benchmark and forecast in the forecast.ss file. Note that in a multi-season model setup, $\text{rel}F_f$ is implemented as $\text{rel}F_{s,f}$ where $s$ is the season. These get multiplied by season duration as they are used.

In the benchmark section of the code, SS3 searches for an Fmult to achieve various management reference points (often referred to as benchmarks). In this search, SS3 calculates a benchmark $F$ as  $F_{ben,f}' = F\text{mult} * \text{rel}F_f$, then calculates equilibrium yield and spawning biomass per recruit (SPR). SS3 searches for the Fmult that satisfies the search conditions, first for user-specified SPR, then for user-specified spawning biomass at a management target (B\textsubscript{TGT} or $F_{0.1}$), then for MSY. The resultant benchmark quantities are reported in the derived quantities, but Fmult and $F_{ben,f}'$ are only reported in the Forecast\_report.sso file. SS3 stores the benchmark Fmult values so that user can invoke them for the forecast.

\myparagraph{Units for Stock Synthesis inputs related to $F$}
Below is a list of items to consider in terms of units for $F$ in SS3:
\begin{itemize}
	\item If F\_ballpark is specified in the control.ss file, its units are the same as annF, so is not fleet-specific.
	
	\item $F$ as parameter values has units of fleet-specific apical $F'$.
	
	\item In the forecast.ss file there is an option to input a vector of relF values. These are dimensionless and will be rescaled to sum to 1.0.
	
	\item In the forecast.ss file there is an option to specify an $F$ scalar for the forecast. The units of $F$ scalar are the same as the Fmult values calculated in benchmark. There are a full set of options for forecast $F$ scalar that can be selected in the forecast file 
	%(-1 = none; 0 = simple; 1 = F\textsubscript{SPR}; 2 = F\textsubscript{MSY} 3 = F\textsubscript{BTGT} or F\textsubscript{0.1}; 4 = Ave F (uses first-last relative F years); and 5 = input annual F scalar). 
	If the forecast $F$ scalar is set as $F_\text{SPR}$, then SS3 will use SPR\_Fmult calculated in benchmark and reported in Forecast-report.sso. If user selects the option to input an annual $F$ scalar, option 5, then the value is input on a following line. Whichever method the user selects for forecast $F$ scalar (Fmult), SS3 will start the forecast by creating a fleet-specific vector of apical $F$ values from Fmult*rel$F_f$.
	
	\item Also in the forecast.ss file, the last section of inputs allows for input of time and fleet specific apical $F_{t,f}'$ values that override the basic forecast $F$ specification described above.
\end{itemize}